\documentclass{article}
\usepackage{graphicx}
\usepackage{filecontents}
\usepackage{gensymb}
\usepackage{graphicx}
\usepackage[T1]{fontenc}
\usepackage[polish]{babel}
\usepackage[utf8]{inputenc}
\usepackage{gensymb}
\usepackage{siunitx}
\usepackage{amsmath}
\graphicspath{ {res/} }
 \title{Wykrywanie żyłek w oku}
\date{2018\\ Maj}
\author{Jakub Tomczak \\ Konrad Kubzdela}
\begin{document}
\maketitle
\clearpage
\section{Wstęp}
Do zrealizowania zadania użyliśmy konwolucyjnej sieci neuronowej z własnym  modelem. Sieć przyjmuje na wejściu kwadratową macierz o zadanym rozmiarze, a na wyjściu zwraca 0 lub 1.
\section{Przygotowanie próbek}
\subsection{Dane}
Do naszego zbioru użyliśmy zdjęć den oczu oraz masek z dwóch różnych stron. Ostatecznie zbiór danych liczył 43 zdjęcia.
\subsection{Próbka}
Pojedynczą próbką jest losowy fragment obrazu oka w skali szarości o zadanym rozmiarze. Wynikiem dla takiej próbki jest wartość środkowego piksela dla zadanego okienka z maski, gdzie 0 oznacza brak żyły a 255 żyłę.  
\section{k-krotna walidacja krzyżowa}

Do oceny naszych danych i modelu użyliśmy 10-krotnej walidacji krzyżowej. Po pobraniu około 500 tysięcy próbek ze wszystkich zdjęć całość mieszaliśmy, a następnie dzieliliśmy na 10 równych części. 
Cały model trenowaliśmy od nowa 10 razy gdzie każda 1/10 z tych danych stanowiła zbiór testowy dla sieci a reszta zbiór do uczenia.

\section{Struktura sieci}
Cała sieć składa się z:
\begin{itemize}
  \item warstwy wejściowej przyjmującej próbkę o rozmiarze okienka
  \item trzech warstw konwolucyjnych
  \item dwóch warstw dokonujących max-pool’ing 
  \item wykorzystującej technikę dropout 
 \item dwóch warstw w pełni połączonych (full connect – fc)
  \item jednej warstwie wyjściowej o rozmiarze 2(żyła lub nie żyła)
\end{itemize}
\section{Błąd średniokwadratowy}
Jako miary jakości użylimy średniej kwadratowej błędów, który obliczyliśmy na podstawie odchyleń wartości pikseli obrazu wyjściowego od wejściowego.



$$\sqrt{\frac{1}{n^2} \sum\limits_{i=1}^n\sum\limits_{j=1}^n (x_{ij}-y_{ij})^2}$$ 
\clearpage
\subsection{Błąd średniokwadratowy w zależności od iteracji}
\centering liczba detektorów n =50\\*
kąt rozwarcia = \ang{50}
\centering

\subsection
 {Błąd średniokwadratowy w funkcji ilości emiterów i detektorów.}
\centering
\subsection{Błąd średniokwadratowy w zależności od zastosowanych filtrów.}
liczba detektorów n=100

\subsection{Błąd średniokwadratowy w zależności od rozpiętości kątowej}
liczba detektorów n =100
\centering

\subsection{Błąd średniokwadratowy w zależności od iteracji}
liczba detektorów n =100
\centering 
\end{document}